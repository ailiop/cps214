A primary benefit of Resilient Overlay Networks (RONs) is their ability to
greatly improve the reliability of Internet packet delivery by detecting
and recovering from outages and path failures more quickly than current
inter-domain routing protocols.  They accomplish this feat at the
application layer by reducing the size of the problem to only include
member nodes.  First generation RONs could support networks of
approximately 50 nodes and recent improvements have increased this size to
approximately 300 nodes.  RON systems have been shown to reduce link outage
during network re-convergence to around 18 seconds instead of several
minutes as can often occur in BGP. Most distributed applications are
sensitive to intermittent network outages, and often are used to carry
sensitive real-time information. Both RON papers discuss VPNs and other
time-sensitive applications, but their experiments only show how RONs
impact route convergence in general terms, without showing the end impact
for specific applications. We seek to confirm the claims of the RON papers
and build a Ruby-based application interface to allow any application to
use a RON for communication, as well as provide a realistic and easy to set
up network environment that can be used as testing grounds for a wide range
of experiments.


%%% Local Variables: 
%%% mode: latex
%%% TeX-master: "project-cps214"
%%% End: 

% LocalWords:  VPN BGP RONs
