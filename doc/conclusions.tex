The results of our Ruby-RON experiment concur with those of both the RON
and Scalable RON projects and show clear benefits for time-sensitive
distributed applications using a functional RON.  Interesting directions
for future research include considering security implications of using a
RON in a true peer-to-peer environment and exploring methods of routing
updates that might avoid the lower bounds involving direct comparisons of
paths.  Additionally, the efficient algorithms used in SRON could be
incorporated into Ruby-RON to provide a highly scalable,
platform-independent architecture.

Furthermore, we feel that the network generator that came out of our work
on this project is an important tool in its own right, as it can greatly
simplify the process of emulating a realistic network environment. We hope
that even in cases where the desired network does not perfectly correspond
to the functionality that is provided by our implementation (e.g.\ one
might need to run IS-IS within the ASes, instead of OSPF), it could still
be easily extensible or modifiable to allow for the extra
functionality. Future work on this part would include consolidating certain
parts of the associated code,\footnote{This is planned to be done soon, in
  the hope that it might actually be useful to the community.} and perhaps
providing greater flexibility in its use.

%%% Local Variables: 
%%% mode: latex
%%% TeX-master: "project-cps214"
%%% End: 


%  LocalWords:  ASes
