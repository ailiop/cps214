Network security, at its core, is a rather dichotomous concept. The sole
purpose of a network is communication; generally, the more open and
trusting a network, the more easily it facilitates this communication. In
contrast, the most secure system is one that is entirely isolated, yet what
benefit can be obtained from such a device? In the end, any networked
system is forced to sacrifice absolute security in order to allow
productivity and growth.

In the past two decades, the Internet has seen exponential growth and has
become the heart of global communications.  This rapid growth has certainly
been facilitated by technological advances but is also attributed to the
open trust nature of the Internet Protocol suite.  Countless tools have
been developed and applied across the Internet to improve security in this
period, but despite all these efforts the growth of the Internet has been
accompanied by parallel growth in cyber-crime.

Hewlett-Packard recently funded a study of 50 large companies and found
that the average cost of cyber-crime per company was approximately \$6
million annually, with at least one successful attack per company per
week~\cite{Ponemon_2011}.  Additionally, a series of high profile attacks
against Sony's PlayStation Network last summer cost approximately \$189
million~\cite{Aamoth_2011}.  In addition to these direct financial costs,
attacks increasingly result in the exposure of privacy data and other
sensitive information.  Stolen privacy data is the driving factor in
growing costs associated with identity
theft~\cite{Schultz_2011}. Hacktivist groups (i.e.\ hackers with a
social/political agenda), such as \emph{Anonymous}, specifically target
such information to undermine companies and groups they find to be at
fault. Their most recent publicized attacks involved a radical neo-Nazi
political group and a mining company in Finland~\cite{Brook_2011}. In both
cases \emph{Anonymous} released individual names and other contact
information encouraging harassment of members of both organizations.

In addition to global financial implications, cyber-attacks are
increasingly being acknowledged as a serious threat by governments around
the world including the US and China. Both countries have recently
activated full military commands to deal with such threats and expect
cyber-warfare to be a reality in the \nth{21} century. As an example, the
Stuxnet virus that disabled Iran's nuclear enrichment centers is widely
speculated to have been developed and placed by foreign government
agents~\cite{Broad_2011}.

These examples clearly show the costly nature of network attacks and
demonstrate the need for effective security.  Unfortunately, these examples
also demonstrate that network attacks continue to occur with security
systems in place. If a method of communication exists, it can be exploited
for unintended use; and for every new security patch applied to a system a
new exploit is discovered. In light of these continually emerging threats,
how can an organization achieve acceptable levels of security?

Virtual Private Networks (VPNs) and network tunneling in general are
crucial elements of such security for modern commerce and government.  They
allow organizations to exchange sensitive information over the Internet and
other public networks with full confidence and in many ways have fueled the
growth of the Internet and the Information Age.  For proof, we need look no
further than Netscape's development of SSL tunneling in 1994 and the
explosive growth of e-commerce and sites such as Amazon.com over the last
two decades.

Unfortunately, the story does not end with VPNs saving the world before
bedtime!  As discussed above, the financial success of e-commerce has
provided plenty of motivation for criminals and like most other security
measures numerous exploits against VPNs, tunneling, and encryption systems
continue to be discovered and patched in a perpetual digital arms race.

One side effect of efforts to secure VPNs is a lowered tolerance to network
instability.  Well-designed modern VPNs quickly kill connections when faced
with link outages or intermittent stability.  This measure is designed to
prevent man-in-the-middle attacks and other forms of eavesdropping or
connection hijacking with the trade-off of artificially reducing
availability.  On a system such as OpenVPN, timeout parameters can be
adjusted to prevent a connection from being killed, but traffic sent during
this period will simply fail to arrive, just as it would without a VPN; as
the timeout interval becomes longer, more effort and care must be put into security.

In many cases this trade-off is merely an annoyance, but for organizations
relying on sensitive real time information (e.g.\ international financial
transactions, military operations, medical evacuations, etc.) every second
counts.  A three minute outage could literally be a matter of life and
death in the latter two cases.

The goal of this project is to incorporate RON and VPN software in a
realistic test environment and determine the exact impact of their combined
use on VPN availability and performance.


%%% Local Variables: 
%%% mode: latex
%%% TeX-master: "project-cps214"
%%% End: 

%  LocalWords:  neo OpenVPN
